%% Generated by Sphinx.
\def\sphinxdocclass{report}
\documentclass[letterpaper,10pt,brazil]{sphinxmanual}
\ifdefined\pdfpxdimen
   \let\sphinxpxdimen\pdfpxdimen\else\newdimen\sphinxpxdimen
\fi \sphinxpxdimen=.75bp\relax

\PassOptionsToPackage{warn}{textcomp}
\usepackage[utf8]{inputenc}
\ifdefined\DeclareUnicodeCharacter
% support both utf8 and utf8x syntaxes
  \ifdefined\DeclareUnicodeCharacterAsOptional
    \def\sphinxDUC#1{\DeclareUnicodeCharacter{"#1}}
  \else
    \let\sphinxDUC\DeclareUnicodeCharacter
  \fi
  \sphinxDUC{00A0}{\nobreakspace}
  \sphinxDUC{2500}{\sphinxunichar{2500}}
  \sphinxDUC{2502}{\sphinxunichar{2502}}
  \sphinxDUC{2514}{\sphinxunichar{2514}}
  \sphinxDUC{251C}{\sphinxunichar{251C}}
  \sphinxDUC{2572}{\textbackslash}
\fi
\usepackage{cmap}
\usepackage[T1]{fontenc}
\usepackage{amsmath,amssymb,amstext}
\usepackage{babel}



\usepackage{times}
\expandafter\ifx\csname T@LGR\endcsname\relax
\else
% LGR was declared as font encoding
  \substitutefont{LGR}{\rmdefault}{cmr}
  \substitutefont{LGR}{\sfdefault}{cmss}
  \substitutefont{LGR}{\ttdefault}{cmtt}
\fi
\expandafter\ifx\csname T@X2\endcsname\relax
  \expandafter\ifx\csname T@T2A\endcsname\relax
  \else
  % T2A was declared as font encoding
    \substitutefont{T2A}{\rmdefault}{cmr}
    \substitutefont{T2A}{\sfdefault}{cmss}
    \substitutefont{T2A}{\ttdefault}{cmtt}
  \fi
\else
% X2 was declared as font encoding
  \substitutefont{X2}{\rmdefault}{cmr}
  \substitutefont{X2}{\sfdefault}{cmss}
  \substitutefont{X2}{\ttdefault}{cmtt}
\fi


\usepackage[Sonny]{fncychap}
\ChNameVar{\Large\normalfont\sffamily}
\ChTitleVar{\Large\normalfont\sffamily}
\usepackage{sphinx}

\fvset{fontsize=\small}
\usepackage{geometry}


% Include hyperref last.
\usepackage{hyperref}
% Fix anchor placement for figures with captions.
\usepackage{hypcap}% it must be loaded after hyperref.
% Set up styles of URL: it should be placed after hyperref.
\urlstyle{same}

\usepackage{sphinxmessages}
\setcounter{tocdepth}{1}



\title{Capacitação IFPA \sphinxhyphen{} Atividades Remotas e a Distância}
\date{14 jul 2020}
\release{1.0.0}
\author{Luiz Fernando Reinoso}
\newcommand{\sphinxlogo}{\vbox{}}
\renewcommand{\releasename}{Release}
\makeindex
\begin{document}

\ifdefined\shorthandoff
  \ifnum\catcode`\=\string=\active\shorthandoff{=}\fi
  \ifnum\catcode`\"=\active\shorthandoff{"}\fi
\fi

\pagestyle{empty}
\sphinxmaketitle
\pagestyle{plain}
\sphinxtableofcontents
\pagestyle{normal}
\phantomsection\label{\detokenize{index::doc}}


O presente momento da história computa a pandemia do novo Coronavírus (COVID\sphinxhyphen{}19), no enfrentamento desta, as instituições de ensino preparam\sphinxhyphen{}se para retomar as aulas com a utilização de ferramentas e tecnologias que forneçam suporte e condições para a execução de aulas, sejam remotas ou a distância.

Somado a isso, os professores podem fazer uso de metodologias de ensino diversas, que podem alavancar ainda mais a ambientação e adaptação dos estudantes, bem como melhorar seu ensino e aprendizagem. Como por exemplo, as metodologias ativas, que colocam o estudate como protagonista de seu ensino e aprendizagem.

Esse contexto cabe\sphinxhyphen{}se no sentido de que certas experiências só podem ser adquiridas de forma isolada,  assim como outras, ocorrem em grupo. Para verficar e se ter ambas visões utilizamos a aprendizagem baseada em problemas, em projetos, por pares, design thinking (conjunto de ideias e insights para abordar problemas, relacionados a futuras aquisições de informações, análise de conhecimento e propostas de soluções) e sala de aula invertida são apenas alguns exemplos.

Neste entorno, o professor se torna um parceiro do estudante, auxiliando o mesmo a se resolver e se encontrar dentro de sua disciplina. Para tanto, o docente parte do que o aluno conhece, usa isso como base de relacionamento entre a ementa a ser trabalhada e ajuda o discente no reconhecimento da mesma, inclusive na formação de estratégias que melhor funcionem para mabas as partes, este processo denomina\sphinxhyphen{}se metacognição.

Fóruns de discussão, chats e tarefas em grupo são os métodos mais conhecidos na EAD, somados aos mecanismos de envio de mensagem isntatanea, videoconferencia, e demais tecnologias, isso mudou a forma tradicioalde se ter aula, mas devemos salientar que a especialidade de cada disciplina e curso pode demandaar de recursos diferenciados, temos jogos digitais, aplicativos web (que não nescessitam de instalação) com recurso de compartilhamento para os mais diversos fins e setores.

Logo, a pesquisa e aprofundamento tencologico deve fazer parte do planejamento docente e educação discente, ou seja, deve\sphinxhyphen{}se observer o tempo de integração e ambientação das tecnologias para os estudantes, leva\sphinxhyphen{}se tempo, então, bom senso e colaboração entre professores pode ajudar, para não termos uso de um emaranhado de tecnologias diferentes, iguais em objetivos, colocando o estudante sobrecarregado de informações e habilidades que este deverá aprender, e em pouco tempo muitas vezes, logo, a interdisciplinaridade deve ser observada, para os recursos serem previamente bem definidos e esclarecidos, envtando\sphinxhyphen{}se perca na produtividade e cosistência do ensino.

Para introduzir os profissionais de educação nesta modalidade, os seguinntes capítulos e informações são colocados a saber:


\chapter{Sobre}
\label{\detokenize{sobre:sobre}}\label{\detokenize{sobre::doc}}
A coordenação de Desenvolvimento de Sistemas tem o prazer de oferecer a todos os docentes e profissionais da educação este treinamento virtual.
O objetivo é fornecer a docentes do Instituto Federal de Educação, Ciência e Tecnologia do Pará (IFPA):
\begin{itemize}
\item {} 
Treinamento no uso de tecnologias pra aulas remotas ou a distância

\item {} 
Maior independência nas atividades de desenvolvimento de material didático e aulas

\item {} 
Inclusão tecnológica

\item {} 
Amparo as atividades docentes

\item {} 
Interatividade para com as práticas síncronas e assíncronas

\end{itemize}

O treinamento e este manual são essenciais para amparar os docentes, portanto, a documentação desenvolvida é uma contrapartida para tornar os docentes mais independentes, assim como realizamos a capacitação de estudantes no SIGAA.

\begin{sphinxadmonition}{note}{Nota:}
Geralmente, o treinamento para o SIGAA é fornecido aos estudantes na semana de adaptação, logo nos primeiros dias de aula.

Caso o estudante falte, o manual do SIGAA para estudantes está disponível so site de seu campus, na sessão “Portal Discente”. O mesmo terá para repor essa carência, bem como aqueles que tem dúvidas e/ou querem rever intrusões de uso.
\end{sphinxadmonition}


\chapter{Metodologias ativas}
\label{\detokenize{metodologias:metodologias-ativas}}\label{\detokenize{metodologias::doc}}
\begin{sphinxShadowBox}
\begin{itemize}
\item {} 
\phantomsection\label{\detokenize{metodologias:id1}}{\hyperref[\detokenize{metodologias:aprednizagem-baseada-em-projetos}]{\sphinxcrossref{Aprednizagem baseada em projetos}}}

\item {} 
\phantomsection\label{\detokenize{metodologias:id2}}{\hyperref[\detokenize{metodologias:aprednizagem-baseada-em-problemas}]{\sphinxcrossref{Aprednizagem baseada em problemas}}}

\item {} 
\phantomsection\label{\detokenize{metodologias:id3}}{\hyperref[\detokenize{metodologias:gamificacao}]{\sphinxcrossref{Gamificação}}}

\item {} 
\phantomsection\label{\detokenize{metodologias:id4}}{\hyperref[\detokenize{metodologias:sala-de-aula-invertida}]{\sphinxcrossref{Sala de aula invertida}}}

\end{itemize}
\end{sphinxShadowBox}

Na sessão introdutória deste manual foi estabelecido os conceitos de metodologias ativas e metacognição, a seguir serão oferecidos, com base em pesquisas acadêmicas, métodos diversos e suas formas de aplicação em sala e ambiente remoto dentro das metodologias ativas.


\section{Aprednizagem baseada em projetos}
\label{\detokenize{metodologias:aprednizagem-baseada-em-projetos}}
Por meio de desafios e conhecimentos dos estudantes é possivel desenvolver muitas soluções, sejam de forma independente ou colaborativa, no uso da Aprednizagem baseada em projetos, buscamos concretizar elementos a partir dos nossos estudos, como:
\begin{itemize}
\item {} 
Produção de vídeos (inclusive de Lives/conferências ao vivo);

\item {} 
Podcast e áudios;

\item {} 
Desenho;

\item {} 
Documentos, como artigos, tutorials, folders, cartazes, entre outros;

\item {} 
Maquetes;

\item {} 
Fotos;

\item {} 
etc…

\end{itemize}

\begin{sphinxadmonition}{note}{Nota:}
Neste modelo, cabe ao professor sempre estar orientando e apresentando os caminhos ao estudante. Logo, ter conhecimento tecnológico dentro dos elementos a serem desenvolvidos é fundamental para dar norteamento.
\end{sphinxadmonition}


\section{Aprednizagem baseada em problemas}
\label{\detokenize{metodologias:aprednizagem-baseada-em-problemas}}
Neste método os sujeitos envolvidos buscam encontrar solução paraum problema conhecido, de forma a aperfeiçoar sua dissolução, ou probema desconhecido, onde através de diversas ações, tentam resolve\sphinxhyphen{}lo, a conquista se da no processo de solução da problemática idnetificada, a aprendizagem baseada em problemas foca na parte teórica da resolução de casos.

Esse método promove a interdisciplinaridade, utiliza\sphinxhyphen{}se as seguintes praticas:
\begin{itemize}
\item {} 
Debates;

\item {} 
Júris;

\item {} 
Discussão em grupo;

\item {} 
etc…

\end{itemize}

\begin{sphinxadmonition}{note}{Nota:}
O resultado são dúvidas, dificuldades e descobertas que são utilizadas para interretação a e apoio a pesquisas e desenvolvimento de diversos tipos de artefatos.
\end{sphinxadmonition}


\section{Gamificação}
\label{\detokenize{metodologias:gamificacao}}
É um método que utiliza elementos comunus a jogos, como desafios, pontuação e niveis. É utilziado em ambientes que enscessitem de engajamento, motivação e processo cirativo. O procefessor gamifica seus processos tradicionais para quebrar a resistÊncia dos estudantes em temas complexos.

Para isso o professor faz uso de:
\begin{itemize}
\item {} 
Dinâmicas atrativas e inteligentes;

\item {} 
Contextos sinalizados;

\item {} 
Jogos;

\item {} 
Ranking;

\item {} 
Desafios;

\item {} 
Premiações;

\item {} 
Ambientes de jogos;

\item {} 
Videos e ferramentas interativas;

\item {} 
etc..

\end{itemize}

\begin{sphinxadmonition}{note}{Nota:}
É nescessário controle da competitividade dos estudantes, incentivo a diversão e colaboração.
\end{sphinxadmonition}


\section{Sala de aula invertida}
\label{\detokenize{metodologias:sala-de-aula-invertida}}
É amplamente difundida nas metodologias ativas, faz uso de tecnologia, principalmente da iternet. Parte das atividades utilizam material ou recursos da internet, inclusive acompanhadas de mecnismos de colaboração em muita das vezes, facilitando a interação de trabalhos em grupo.

Para que uma sala de aula invertida funcione, os alunos devem se comprometer de fato aos desafios, desde a ambientação a sua difusão e uso. Muitas das vezes, práticas presenciais se apoiam nas online, e vice versa.

Utilizam\sphinxhyphen{}se na sala de aula invertida, todo tipo de tecnologia sincrona e assincrona, as mais comuns são:
\begin{itemize}
\item {} 
Fóruns;

\item {} 
Chats;

\item {} 
Comentários;

\item {} 
Troca de mensagens e documentos de forma instântanea;

\item {} 
Edição de documentos de forma colaborativa;

\item {} 
Envio e leitura de emails;

\item {} 
Webconferências;

\item {} 
Uso de recusos virtuais com compartilhamento de dados;

\item {} 
Redes sociais;

\item {} 
etc….

\end{itemize}


\chapter{índice}
\label{\detokenize{index:indice}}\begin{itemize}
\item {} 
\DUrole{xref,std,std-ref}{genindex}

\item {} 
\DUrole{xref,std,std-ref}{modindex}

\item {} 
\DUrole{xref,std,std-ref}{search}

\end{itemize}



\renewcommand{\indexname}{Índice}
\printindex
\end{document}